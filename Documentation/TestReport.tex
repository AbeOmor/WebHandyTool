\documentclass[12pt, titlepage]{article}
\usepackage{xcolor} % for different colour comments
\usepackage[left=15mm,right=15mm,top=1in,bottom=1in]{geometry}
\usepackage{framed}
\usepackage{graphicx}
\graphicspath{ {images/} }
%% Comments
\newif\ifcomments\commentsfalse %i replaced comments true by comment false so the comments will be hidden

\ifcomments
\newcommand{\authornote}[3]{\textcolor{#1}{[#3 ---#2]}}
\newcommand{\todo}[1]{\textcolor{red}{[TODO: #1]}}
\else
\newcommand{\authornote}[3]{}
\newcommand{\todo}[1]{}
\fi

\newcommand{\wss}[1]{\authornote{magenta}{SS}{#1}}
\newcommand{\hm}[1]{\authornote{blue}{HM}{#1}} %Hediyeh
\newcommand{\tz}[1]{\authornote{blue}{TZ}{#1}} %Tahereh
\newcommand{\pl}[1]{\authornote{blue}{PL}{#1}} %Peng

\begin{document}
\title{pylinkvalidator \\
 Test Report : newAGEtech, Group H }
\author{Genevieve Okon (Okong), Abraham Omorogbe(Omorogoa),\\
 Eric Le Forti(Leforte)}
\date{\today}
\maketitle

\tableofcontents
\pagebreak

\begin{center}\textbf{Revision History}\end{center}
\begin{table}[h!]
\centering
	\begin{tabular}{| p{1.5cm} | p{2.5cm} | p{7cm} |p{3cm} |}    \hline
	Revision  &Revision Date &Description of Change &Author\\ \hline
	1& 20-10-15 &Initiate Test Plan Document &Eric Le Fort\\ \hline
       \end{tabular}
       \caption{Revision History}
       \label{table:Revision History}
\end{table}

\listoftables
\listoffigures
\pagebreak



\section{Introduction}
\subsection{Objective}
The purpose of this report is to specify the methodology of testing to be used for Pylinkvalidator in detail. Every test case will be accompanied by a short description to convey the reason each test was written as well as a breakdown of expected results as well as whether those results were achieved or not. Following that section the document will trace the tests back to the requirements and then provide a more general summary of the results of testing.

\subsection{Approach}
The methodology to be used for testing will involve a succinct set of tests to prove each requirement is fully functional and performing at an acceptable level. These tests will cover white-boxed boundary cases, cases dealing with extremes as well as standard cases.\\

Certain tests, such as those concerning usability or involving acquiring user input, are tested much more straightforwardly using manual methods. Therefore, these sorts of tests will be performed in a manual manner. All other tests will be conducted using automated testing utilizing a testing suite known as PyUnit.

\subsection{Tables of Acronyms, Abbreviations \& Definitions}

\begin{table}[h!]
\centering
\begin{tabular}{| p{5cm} | p{5cm} |}    \hline
	Term &Meaning\\ \hline
\end{tabular}
\caption{Acronyms \& Abbreviations}
\label{table:Acronyms and Abbreviations}
\end{table}

\begin{table}[h!]
\centering
\begin{tabular}{| p{5cm} | p{5cm} |}    \hline
	Term &Definition\\ \hline
	PyUnit &A widely accepted testing suite to be used with the Python programming language.\\ \hline
\end{tabular}
\caption{Definitions}
\label{table:Definitions}
\end{table}



\section{Functional System Tests}
\subsection{F1: Exact String Searching}
//TODO fill in test cases
\begin{table}[h!]
\centering
\begin{tabular}{| p{6cm} | p{5cm} | p{5cm} |}    \hline
	Process &Expected Results &Actual Results\\ \hline
\end{tabular}
\caption{F1 Tests}
\label{table:F1 Tests}
\end{table}

\subsection{F2: Similar String Searching}
//TODO fill in test cases
\begin{table}[h!]
\centering
\begin{tabular}{| p{6cm} | p{5cm} | p{5cm} |}    \hline
	Process &Expected Results &Actual Results\\ \hline
\end{tabular}
\caption{F2 Test Cases}
\label{table:F2 Test Cases}
\end{table}

\subsection{F3: }
//TODO fill in test cases
\begin{table}[h!]
\centering
\begin{tabular}{| p{6cm} | p{5cm} | p{5cm} |}    \hline
	Process &Expected Results &Actual Results\\ \hline
\end{tabular}
\caption{F3 Test Cases}
\label{table:F3 Test Cases}
\end{table}

//TODO write tests and determine sections

\section{Non-Functional Tests}
\subsection{Usability}
//TODO list as NF1:, NF2:,..., NFn
//TODO
\subsection{Performance}
//TODO
\subsection{Robustness}
Implicitly included within the list of functional test cases since extreme cases as well as boundary cases will be performed.

\section{Traceability to Requirements}
\begin{table}[h!]
\centering
\begin{tabular}{| p{2.5cm} | p{8cm} |}    \hline
	Requirement Designation &Associated Requirement\\ \hline
	F1 &Functional requirements 5 and 7.\\ \hline
	F2 &Functional requirements 5 and 7.\\ \hline
\end{tabular}
\caption{F1 Tests}
\label{table:F1 Tests}
\end{table}
//TODO ensure above table is fully completed after completion of test case specifications.

\section{Testing Summary}
\subsection{Code Coverage}
\subsection{Testing Results}
\subsection{Changes Due to Testing}

\noindent \wss{This is an example comment.  You can turn comments off by replacing
  commentstrue by commentsfalse.}\\
\hm{Sample comment}\\
\tz{Sample comment}\\
\pl{Sample comment}

\end{document}
